\documentclass[11pt,a4paper,sans]{moderncv}

\moderncvstyle{banking}                            % style options are 'casual' (default), 'classic', 'oldstyle' and 'banking'
\moderncvcolor{blue}                                % color options 'blue' (default), 'orange', 'green', 'red', 'purple', 'grey' and 'black'
\usepackage[utf8]{inputenc}                       % if you are not using xelatex ou lualatex, replace by the encoding you are using
\usepackage[scale=0.75]{geometry}
\usepackage{import}

% personal data
\name{Sagnika}{Dasgupta}
\title{}                               % optional, remove / comment the line if not wanted
\address{1263, 2nd Cross, 2nd Main, HAL 3rd Stage, New Tippasandra,Bangalore 560075}{}{}% optional, remove / comment the line if not wanted; the "postcode city" and and "country" arguments can be omitted or provided empty
\phone[mobile]{+91 9538766397}                   % optional, remove / comment the line if not wanted
\email{sagnika.dasgupta29@gmail.com}                               % optional, remove / comment the line if not wanted

\begin{document}
%\begin{CJK*}{UTF8}{gbsn}                          % to typeset your resume in Chinese using CJK
%-----       resume       ---------------------------------------------------------
\makecvtitle

\small{Currently completing final semester of Masters in Biotechnology. Previously Masters in Environmental Management. Previous experience in event and project management in environment related issues. Work well in teams and projects with multiple stakeholders.}

\section{Previous Employment}

\vspace{6pt}

\begin{itemize}

\item{\cventry{September 2013--December 2015}{Project Associate}{Centre for Environment Education}{Kolkata}{}{\vspace{3pt}}}
    \textbf{Job Responsibilities}
    \begin{itemize}
    \item{ Developing proposal for Municipal Solid Waste Management.}
    \item {Increasing Evironment Awareness in Schools}
    \item {Networking with experts and professors for guidance and involvement of new techniques for use of fly ash in agriculture field land reclamation and agriculture in experimental basis and overall management of fly ash}
    \item Documentation and reporting.

    \end{itemize}

\vspace{6pt}

\item{\cventry{ January 2015-- August 2015}{Event Coordinator}{Daily Dump}{Bangalore}{}{\vspace{3pt}}}

\end{itemize}

\subsection{Notable Projects}
\begin{itemize}
\item{\textbf{Centre for Environment Education (September, 2013 – August 2014)}: \textit{'Nabadiganta Project on Fly Ash Management and Community Development'}

      \vspace{3pt}

The project aimed at maximizing utilization of the fly ash produced by thermal power plant and providing green belt all around the plant
for maintaining ecological balance. Special emphasis on land use and Bio-diversity by way of development of green belts, energy plantations,
reclamation of abandoned ash pond and ecological monitoring in the project areas is also the part of the project}

Key Responsibilities:
    \begin{itemize}
        \item Interacting with stakeholders to analyses the gaps in the prevailing system.
        \item Performing detailed analysis, analyses research findings, draw inferences and generate paper on the basis of data.
        \item Conducting resource mapping, PRA, Baseline survey and awareness programme.
        \item Overall project coordination.
        \item Administrative work related to the project.
    \end{itemize}
\end{itemize}

\vspace{5pt}

\begin{itemize}
      \newpage
\item{\textbf{Centre for Environment Education (September, 2014 – November 2015)} \textit{'Urja Chetna Project '}

\vspace{3pt}

\small{ The Urja Chetana Project aimed to generate awareness and action on energy conservation, waste management, rainwater harvesting and medicinal
plant garden practices amongst students and individuals, who would carry the learnings to their communities.
The strategy is to implement the programme in schools through eco-clubs/energy clubs.
More details of the project can be found aat http://www.urjachetana.in }}
Key Responsibilities:
\begin{itemize}
\item Enabling Eco-clubs and nature clubs of 30 schools in and around Kolkata to carry out waste management and energy conservation programmers (composting, paper recycling)
\item Conducting various awareness programmes and activities in schools. (competitions, film and quiz shows, energy and waste audit)
\item Conducting community outreach programme with the students.
\item Organizing training programmes, Workshops, Exposure trips for students and teachers.
\item Facilitating Rainwater Harvesting Installation in schools.
\item Conducting Surveys , qualitative and quantitative analysis on waste generation and its management in schools.
\end{itemize}

\vspace{3pt}
\item Worked on \textbf{Bengali Trans-adaptation of Science Express Climate Actions Special (SECAS)} documents
\item Organized \textbf{Urban Climate Change Resilience Project} with NIUA
\item Organized \textbf{Sankalp Diwas} 2014, an initiation of CEE in collaboration with \textbf{National Clean Ganga Mission (NCGM)} at Farakka, West Bengal.
\item Worked on documentation on good pratices on marine and coastal biodiversiy conservation in India, a Indo-German project Conservation and Sustainable Management of Existing and
Potential Coastal and Marine Protected Areas (CMPA)
\item Organized and coordinated Paryavaran Mitra Puraskar Award event 2013-14. CEE, Kolkata.
\end{itemize}
\section{Education}

\vspace{5pt}

\subsection{Academic Qualifications}

\vspace{5pt}

\begin{itemize}

\item{\cventry{2016--}{M.Sc. in BioTechnology}{Garden City College}{Bangalore}{\textit{}}{}}

\item{\cventry{2011--2013}{M.Sc. in Environmental Management }{University of Kalyani}{Kalyani, WB}{\textit{}}{}}

\item{\cventry{2007--2011}{B.Sc. in Zoology(Honours)}{Vivekananda College for Women}{Kolkata}{\textit{}}{}}  % arguments 3 to 6 can be left empty

\item{\cventry{2005--2007}{WBCHSE(10+2)}{Sir Ramesh Mitter Girls High School}{Kolkata}{\textit{}}{}}
\item{\cventry{2005--2007}{WBBSE(10)}{Sri Sarada Ashram Balika Vidyalaya}{Kolkata}{\textit{}}{}}
\end{itemize}

\vspace{2pt}

\subsection{Notable Projects}

\vspace{5pt}

\begin{itemize}

\item{\textbf{Masters(BioTechnology) Project (Planned for currentt semester):} \textit{'Green Synthesis and characterization of Selenium Nanoparticles and its Therapeutic Effects'}

\vspace{3pt}

\small{}}

\vspace{3pt}

\item{\textbf{Masters(Environmental Management) Project} \textit{'Cultivation of Chlorella in live stock waste water for nutrient reclamation'}

\vspace{3pt}

\small{}}




\end{itemize}

\section{Technical and Personal skills}

\vspace{6pt}

\begin{itemize}

\item \textbf{Vermicomposting}
\vspace{6pt}
\item \textbf{Computer Proficiency:} MS-Word, MS-Excel, MS-Powerpoint. Basics of Nebcutter, Primer9, ClustalOmega

\vspace{6pt}


\item \textbf{General Business Skills:} Good presentation skills, Works well in a team. Experienced Event manager
\vspace{6pt}
\item \textbf{Linguistic Profeciency}: English, Bengali, Hindi
\end{itemize}
\end{document}


%% end of file `template.tex'.

